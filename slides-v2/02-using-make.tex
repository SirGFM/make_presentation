\section{Using (GNU) Make}

\begin{frame}{\secname}
    Make defines a couple of variable that may be used to modify its built-in implicit rules:

    \begin{itemize}
        \item \texttt{CC}: Program for compiling C programs; default ‘cc’;
        \item \texttt{CXX}: Program for compiling C++ programs; default ‘g++’;
        \item \texttt{CFLAGS}: Extra flags to give to the C compiler;
        \item \texttt{CXXFLAGS}: Extra flags to give to the C++ compiler;
        \item \texttt{LDFLAGS}: Extra flags to give to the linker (e.g.: \texttt{-L /lib/path});
        \item \texttt{LDLIBS}: Libraries flags (e.g., \texttt{-Wl,-rpath}) or name (e.g., \texttt{-lcrypto}) to give to the linker.
    \end{itemize}

    For a non-exhaustive, but more complete list, see: \\~\\

    https://www.gnu.org/software/make/manual/html\_node/Implicit-Variables.html\#Implicit-Variables
\end{frame}

\begin{frame}{\secname}
    \begin{alertblock}{Attention!}
        Use these predefined variables (or similars) whenever possible! \\~\\
    \end{alertblock}

    Cross-compiling to another architecture may be achieved
    by (almost) as little as appending \texttt{-m<arch>} and
    \texttt{--sysroot=<...>} to \texttt{CFLAGS} (or \texttt{CC}). \\~\\

    Also, the entire tool-chain may be overriden by setting
    \texttt{CC}, \texttt{AR} etc (e.g., using MinGW to
    cross-compile to Windows).
\end{frame}

\defverbatim[colored]\makeTarget{
    \begin{lstlisting}[basicstyle=\tiny, language=make]
linux32: bin/linux32_release/my-program.app
linux32_debug: bin/linux32_debug/my-program.app
win32: bin/win32_release/my-program.app
win32_debug: bin/win32_debug/my-program.app

%.app: $(OBJS)
	$(CC) $(CFLAGS) $(LDFLAGS) -o $@ $(OBJS) $(LDLIBS)
    \end{lstlisting}
}

\begin{frame}{\secname}
    Make accepts \textbf{phony} targets, i.e. targets without a matching artifact. \\~\\

    Those are useful for adding targets with more friendly, and generic, names: \\~\\

    \makeTarget

\end{frame}

\defverbatim[colored]\makePhony{
    \begin{lstlisting}[basicstyle=\tiny, language=make]
.PHONY: all clean
clean:
	$(RM) $(OBJS) $(TARGET)
    \end{lstlisting}
}

\begin{frame}{\secname}
    If no rule is specified, Make tries to build the first target (usually the phony target \texttt{all}). \\~\\

    Another common phony target is \texttt{clean}. \\~\\

    However, if a file were created with the name of any of the phony targets, those wouldn't be called anymore (since the file would be considered up to date). \\~\\

    To avoid that, the special target \texttt{.PHONY} may be used: \\~\\

    \makePhony

\end{frame}
