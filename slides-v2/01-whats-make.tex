\section{What's Make}

\begin{frame}{\secname}
    Make is a build tool that controls the generation of "artifacts" from resources

    \begin{itemize}
        \item Originally written by Stuart Feldman, to manage building programs;
        \item Can actually be used to build anything;
        \item Decides whether a given \textbf{target} must be updated based on the modification time of the file itself and of its \textbf{prerequisites};
        \item A target may have a custom building \textbf{rule} or use an \textbf{implicit rule}.
    \end{itemize}

    Make comes in various flavors:

    GNU Make (gmake), BSD Make (bmake), Microsoft Program Maintenance Utility (NMAKE.EXE)
\end{frame}

\defverbatim[colored]\makeMain{
    \begin{lstlisting}[basicstyle=\tiny, language=Makefile]
main: main.c
	gcc -o main main.c
    \end{lstlisting}
}

\defverbatim[colored]\makeSyntax1{
    \begin{lstlisting}[basicstyle=\tiny, language=Makefile]
target: prereq0 prereq1 prereq2 ... prereqN
	<build-command-0>
	<build-command-1>
	...
	<build-command-M>
    \end{lstlisting}
}

\section{Syntax overview}
\begin{frame}{\secname}
    The following is a valid Makefile for building main from main.c: \\~\\

    \makeMain

    Which could be broken down as: \\~\\

    \makeSyntax

\end{frame}

\defverbatim[colored]\makeImplicit{
    \begin{lstlisting}[basicstyle=\tiny, language=Makefile]
\%.o: \%.c
	gcc -c -o \$@ \$<
    \end{lstlisting}
}

\defverbatim[colored]\makeImplicitCFile{
    \begin{lstlisting}[basicstyle=\tiny, language=Makefile]
\%.o: \%.c
	\$(CC) -c \$(CFLAGS) \$(CPPFLAGS) -o \$@ \$<
    \end{lstlisting}
}

\begin{frame}{\secname: \small\subsecname\normalsize}
    Rules can be defined implicitly, so they may be used for any number of targets. The following rule bellow matches:

    \makeImplicit

    \begin{itemize}
        \item main.o: main.c
        \item foo.o: foo.c
        \item ...
    \end{itemize}

    Since Make was written to build programs, it has a couple of built-in implicit rules. For example:

    \makeImplicitCFile

\end{frame}

\begin{frame}{\secname: \small\subsecname\normalsize}
    (GNU) Make's implicit rules can be listed by running \texttt{make -p}.

    Fun fact: because of these implicit rules, you can run \texttt{make main} in a directory with only a \texttt{main.c} and it will correctly compile and link the program.

\end{frame}
