\section{What's GNU Make}

\begin{frame}{\secname}
    \begin{itemize}
        \item Tool that controls generation of "artifacts" from resources
        % TODO Further this, as there may be issues
        \item Keeps track of what's been changed, minimizing build time
        \item Automatically handles building well known "artifacts"
        \item Allows creating new rules for building unknown artifacts
    \end{itemize}
\end{frame}

\subsection{Syntax overview}
\begin{frame}[fragile]
    \frametitle{\secname: \small\subsecname\normalsize}

    The following is a valid Makefile for building main from main.c: \\~\\

    \begin{lstlisting}[language=make]
main: main.c
    gcc -o main main.c
    \end{lstlisting}

\end{frame}

\begin{frame}[fragile]
    \frametitle{\secname: \small\subsecname\normalsize}

    Basically, Make has the following syntax: \\~\\

    \begin{lstlisting}[language=make]
# NOTE: These must be indented by tabs!
target: prereq1 prereq2 ... prereqN
    <build-command>
    \end{lstlisting}
\end{frame}

\subsection{A simple Makefile (demo01-2)}
\begin{frame}[fragile]
    \frametitle{\secname: \small\subsecname\normalsize}

    If the previous example were written as a Makefile, it could look like the following:

    \begin{lstlisting}[language=make]
demo01-2: demo01-2_src/main.o demo01-2_src/data.o
    gcc -o demo01-2 demo01-2_src/main.o \
        demo01-2_src/data.o

demo01-2_src/main.o: demo01_src/main.c
    gcc -o demo01-2_src/main.o \
        -c demo01_src/main.c

demo01-2_src/data.o: demo01_src/data.c
    gcc -o demo01-2_src/data.o \
        -c demo01_src/data.c
    \end{lstlisting}
\end{frame}

\begin{frame}{\secname: \small\subsecname\normalsize}
    Let's go live!

    % This is pretty much the previous example all over again
\end{frame}

\subsection{Keeping it simple}
\begin{frame}{\secname: \small\subsecname\normalsize}
    \begin{itemize}
        \item Make was primarily written to build programs
        \item Any C program builds pretty similarly
        \item (the same is valid for other languages as well!)
    \end{itemize}
\end{frame}

\begin{frame}[fragile]
    \frametitle{\secname: \small\subsecname (part 1)\normalsize}

    For example, Make already knows how to build a .o from a .c: \\~\\

    \begin{lstlisting}[language=make]
demo01-2: demo01-2_src/main.o demo01-2_src/data.o
     gcc -p demo01-2 demo01-2_src/main.o \
        demo01-2_src/data.o

demo01-2_src/main.o: demo01_src/main.c

demo01-2_src/data.o: demo01_src/data.c
    \end{lstlisting}
\end{frame}

\begin{frame}[fragile]
    \frametitle{\secname: \small\subsecname (part 2)\normalsize}

    However, Make does not know how to build a binary from multiple sources:

    \begin{lstlisting}[language=make]
demo01-3: demo01-3_src/main.o demo01-3_src/data.o
    \end{lstlisting}

    Unless the target shares the name of one of its targets:
    \begin{lstlisting}[language=make]
demo01-3_src/main: demo01-3_src/main.o demo01-3_src/data.o
    \end{lstlisting}
\end{frame}

% As much as I find this example funny, it doesn't really seem
% helpful... D:
%
%\begin{frame}{\secname: \small\subsecname\normalsize}
%    Question: How long is the smallest Makefile? \\~\\ \pause
%
%    Let's write it live! (demo02)
%
%    % make demo02_src/main
%\end{frame}

\subsection{Keeping it organized}
\begin{frame}{\secname: \small\subsecname\normalsize}
    For more complex programs, you may set some predefined
    variables: \\~\\

    \begin{itemize}
        \item \texttt{CC}: The default compiler for C files
        \begin{itemize}
            \item Use \texttt{CXX} for C++
        \end{itemize}
        \item \texttt{CFLAGS}: variables supplied when building
        \begin{itemize}
            \item Use \texttt{CXXFLAGS} for C++
        \end{itemize}
        \item \texttt{LDFLAGS}: Path to custom libraries
        \item \texttt{LDLIBS}: Libraries required by the program
    \end{itemize}
\end{frame}

\begin{frame}{\secname: \small\subsecname\normalsize}
    Example: demo02
\end{frame}

\begin{frame}{\secname: \small\subsecname\normalsize}
    \begin{alertblock}{Attention!}
        Use these predefined variables (or similars) whenever possible!
    \end{alertblock}

    \begin{itemize}
        \item Cross-compiling to another architecture may be achieved
        by "simply" appending \texttt{-m<arch>} and
        \texttt{--sysroot=<...>} to \texttt{CFLAGS} (or \texttt{CC})

        \item The entire tool-chain may be overriden by setting
        \texttt{CC}, \texttt{AR} etc (e.g., using MinGW to
        cross-compile to Windows)
    \end{itemize}
\end{frame}
