\section{What's GNU Make}

\begin{frame}{\secname}
    \begin{itemize}
        \item Tool that controls generation of "artifacts" from resources
        % TODO Further this, as there may be issues
        \item Keeps track of what's been changed, minimizing build time
        \item Automatically handles building well known "artifacts"
        \item Allows creating new rules for building unknown artifacts
    \end{itemize}
\end{frame}

\subsection{Syntax overview}
\begin{frame}[fragile]
    \frametitle{\secname: \small\subsecname\normalsize}

    The following is a valid Makefile for building main from main.c: \\~\\

    \begin{lstlisting}[language=make]
main: main.c
    gcc -o main main.c
    \end{lstlisting}

\end{frame}

\begin{frame}[fragile]
    \frametitle{\secname: \small\subsecname\normalsize}

    Basically, Make has the following syntax*: \\~\\

    \begin{lstlisting}[language=make]
# NOTE: These must be indented by tabs!
target: prereq1 prereq2 ... prereqN
    <build-command>
    \end{lstlisting}

    \small * Obviously, this ignores a lot of features
\end{frame}

\subsection{A simple Makefile (demo01-2)}
\begin{frame}[fragile]
    \frametitle{\secname: \small\subsecname\normalsize}

    \begin{lstlisting}[language=make]
demo01-2: demo01-2/main.o demo01-2/data.o
    gcc -o demo01-2 demo01-2/main.o \
        demo01-2/data.o

demo01-2/main.o: demo01-2/main.c
    gcc -o demo01-2/main.o -c demo01-2/main.c

demo01-2/data.o: demo01-2/data.c
    gcc -o demo01-2/data.o -c demo01-2/data.c
    \end{lstlisting}
\end{frame}

\begin{frame}{\secname: \small\subsecname\normalsize}
    Let's go live!

    % This is pretty much the previous example all over again
\end{frame}

\subsection{Keeping it simple}
\begin{frame}{\secname: \small\subsecname\normalsize}
    \begin{itemize}
        \item Make was primarily written to build programs
        \item Any C program builds pretty similarly \\~\\ \pause
        \item Make already has some built-in rules
    \end{itemize}
\end{frame}

\begin{frame}[fragile]
    \frametitle{\secname: \small\subsecname (part 1)\normalsize}

    Make already knows how to build a .o from a .c: \\~\\

    \begin{lstlisting}[language=make]
demo01-2: demo01-2/main.o demo01-2/data.o

demo01-2/main.o: demo01-2/main.c

demo01-2/data.o: demo01-2/data.c
    \end{lstlisting}
\end{frame}

\begin{frame}[fragile]
    \frametitle{\secname: \small\subsecname (part 2)\normalsize}

    More than that; Make knows how to find some dependecies (e.g., that a .o may be built from its .c): \\~\\

    \begin{lstlisting}[language=make]
demo01-2: demo01-2/main.o demo01-2/data.o
    \end{lstlisting}
\end{frame}

\begin{frame}{\secname: \small\subsecname\normalsize}
    Question: How long is the smallest Makefile? \\~\\ \pause

    Let's write it live! (demo02)

    % make demo02_src/main
\end{frame}
