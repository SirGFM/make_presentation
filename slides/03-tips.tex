\section{GNU Make tips}

\subsection{Variable attribution}
\begin{frame}{\secname: \small\subsecname\normalsize}
    Attributing a value to a variable is as simples as doing:

    \begin{itemize}
        \item \texttt{VAR := value}
    \end{itemize}

    or

    \begin{itemize}
        \item \texttt{VAR = value}
    \end{itemize}
\end{frame}

\begin{frame}{\secname: \small\subsecname\normalsize}
    \begin{block}{Variable evaluation!}
        Diffence between \texttt{VAR := value} and \texttt{VAR = value}
    \end{block}

    \begin{itemize}
        \item \texttt{VAR := value} uses immediate evaluation
        \begin{itemize}
            \item \texttt{VAR := \$(VAR) some-value} is OK
            \item \texttt{VAR += some-value} is also OK (but adds a space)
        \end{itemize}
        \item \texttt{VAR = value} uses lazy evaluation
        \begin{itemize}
            \item \texttt{VAR = \$(VAR) some-value} is \textbf{NOT} OK
            \item \texttt{VAR += some-value} is OK
        \end{itemize}
    \end{itemize}

    See demo03\_src!

\end{frame}
