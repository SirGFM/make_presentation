\section{GNU Make tips}

\subsection{Variable attribution}
\begin{frame}{\secname: \small\subsecname\normalsize}
    Attributing a value to a variable is as simples as doing:

    \begin{itemize}
        \item \texttt{VAR := value}
    \end{itemize}

    or

    \begin{itemize}
        \item \texttt{VAR = value}
    \end{itemize}
\end{frame}

\begin{frame}{\secname: \small\subsecname\normalsize}
    \begin{block}{Variable evaluation!}
        Diffence between \texttt{VAR := value} and \texttt{VAR = value}
    \end{block}

    \begin{itemize}
        \item \texttt{VAR := value} uses immediate evaluation
        \begin{itemize}
            \item \texttt{VAR := \$(VAR) some-value} is OK
            \item \texttt{VAR += some-value} is also OK (but adds a space)
        \end{itemize}
        \item \texttt{VAR = value} uses lazy evaluation
        \begin{itemize}
            \item \texttt{VAR = \$(VAR) some-value} is \textbf{NOT} OK
            \item \texttt{VAR += some-value} is OK
        \end{itemize}
    \end{itemize}

    See demo03\_src!

\end{frame}

\subsection{Variable substitution}
\begin{frame}[fragile]
    \frametitle{\secname: \small\subsecname\normalsize}

    Variable substitution may happen when accessing a variable: \\~\\
    \begin{lstlisting}[language=make]
SOURCES := main.c data.c lib.c
OBJS := $(SOURCES:.c=.o)
    \end{lstlisting}

    and on rules themselves: \\~\\
    \begin{lstlisting}[language=make]
%.o: %.c
    \end{lstlisting}

\end{frame}

\subsection{Creating directories for targets}
\begin{frame}[fragile]
    \frametitle{\secname: \small\subsecname\normalsize}

    By defining every rule and using variable substitution, it's possible
    to generate every output directory: \\~\\

    \begin{lstlisting}[language=make]
OBJS := obj/main.o obj/subdir/helper.o

bin/main: $(OBJS) | bin/.mkdir

obj/%.o: src/%.c | obj/%.mkdir

# $(@D) retrieves the directory of the current target
%.mkdir:
    mkdir -p $(@D)
    \end{lstlisting}

    See demo04\_src!
\end{frame}
